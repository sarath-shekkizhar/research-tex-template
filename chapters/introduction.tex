\vspace{-15pt}
\section{Introduction}
% ========================================
% INTRODUCTION TEMPLATE
% ========================================
% A good introduction typically follows this structure:
% 1. Motivation & Context (1-2 paragraphs)
% 2. Problem Statement (1 paragraph)
% 3. Limitations of Existing Work (1 paragraph)
% 4. Your Approach Overview (1-2 paragraphs)
% 5. Key Contributions (1 paragraph, often bulleted)
% 6. Paper Organization (optional, 1 sentence)

% Paragraph 1: Motivation & Context
% Start broad - what is the general area and why is it important?
% Provide context for readers who may not be domain experts.
% Example: "Recent advances in [field] have enabled [capability], opening new possibilities for [application]."

[Motivation Paragraph] Describe the broader context and importance of your research area. Why should readers care about this problem domain? What recent developments or trends make this timely?

% Paragraph 2: Problem Statement
% Narrow down to the specific problem you're addressing.
% Make it clear what gap exists in current solutions.
% Example: "However, existing approaches face challenges when [specific scenario], limiting their applicability to [real-world use case]."

[Problem Statement] Identify the specific problem or limitation that your work addresses. What challenges do current methods face? Why is this problem difficult or important to solve?

% Paragraph 3: Limitations of Prior Work
% Discuss why existing approaches are insufficient.
% Be respectful but clear about what's missing.
% Cite relevant prior work appropriately.

[Prior Work Limitations] Explain what existing methods have tried and why they fall short. What key insight or capability is missing? Reference relevant prior work using \texttt{\textbackslash{}cite\{\}} commands.

% Paragraph 4-5: Your Approach
% Introduce your method/framework with a clear name.
% Explain the key insight or innovation that enables your approach.
% Reference figures if you have overview diagrams (e.g., Figure~\ref{fig:overview}).
% Describe the main technical components at a high level.

[Your Approach] Introduce your method or framework. Give it a clear, memorable name and explain the key innovation that distinguishes it from prior work. Describe the main components or stages of your approach. If you have an overview figure, reference it here.

% Paragraph 6: Key Contributions
% Clearly enumerate your contributions, often as a bulleted list.
% Be specific and concrete about what's novel.

[Contributions Summary] We make the following key contributions:
\begin{itemize}
    \item \textbf{Contribution 1:} A novel method/framework/algorithm for [specific task]. Briefly explain what makes it unique.
    \item \textbf{Contribution 2:} A new benchmark/dataset/evaluation for [specific purpose]. Describe its key characteristics.
    \item \textbf{Contribution 3:} Comprehensive experiments showing [X\% improvement] on [specific metrics] compared to baselines. Highlight key empirical findings.
    \item \textbf{Contribution 4 (optional):} Additional contributions such as theoretical analysis, user studies, open-source releases, etc.
\end{itemize}

% Optional: Paper Organization
% Some papers include a brief roadmap, others don't. Use your judgment.
% Example: "The rest of this paper is organized as follows: Section~\ref{sec:method} describes our approach, Section~\ref{sec:experiments} presents experimental results, and Section~\ref{sec:related} discusses related work."

% [Optional Paper Roadmap] The remainder of this paper is organized as follows: Section~\ref{sec:method} describes our method in detail, Section~\ref{sec:experiments} presents experimental results, Section~\ref{sec:related} discusses related work, and Section~\ref{sec:conclusion} concludes.
