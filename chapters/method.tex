
\section{Method}
\label{sec:method}
% ========================================
% METHOD SECTION TEMPLATE
% ========================================
% This section describes your approach in detail.
% Structure suggestions:
% 1. Problem Formulation (optional separate section or subsection)
% 2. Overview of your approach
% 3. Detailed description of each component
% 4. Algorithm pseudocode (if applicable)
% 5. Theoretical analysis (if applicable)

% Optional: Separate problem formulation section
\subsection{Problem Formulation}
\label{sec:formulation}

[Problem Setup] Define the problem formally. Introduce necessary notation and mathematical definitions. What are the inputs and outputs? What are the constraints or assumptions?

% Example notation:
% - Let $\mathcal{D} = \{(x_i, y_i)\}_{i=1}^N$ denote the dataset
% - Given input $x \in \mathcal{X}$, we aim to predict $y \in \mathcal{Y}$
% - Define the objective function as $\mathcal{L}(\theta) = ...$

\subsection{Method Overview}
\label{sec:method-overview}

[High-Level Description] Provide an overview of your approach before diving into details. Explain the key intuition and main steps. Reference any overview figures you have.

% Example: "Our method consists of three main stages: (1) preprocessing, (2) model training, and (3) inference optimization. Figure~\ref{fig:method} illustrates the complete pipeline."

\subsection{Component 1: [Name]}
\label{sec:component1}

[Detailed Component Description] Describe the first major component of your method. Include:
- Mathematical formulation
- Design choices and rationale
- Connections to prior work
- Any implementation details crucial for understanding

% Example subsections you might include:
% - Architecture design
% - Loss function
% - Training procedure
% - Optimization details

\subsection{Component 2: [Name]}
\label{sec:component2}

[Second Component] Describe additional components similarly. Break down complex methods into digestible parts.

\subsection{Algorithm}
\label{sec:algorithm}

[Pseudocode] If applicable, include algorithm pseudocode for clarity. Use the algorithm/algorithmic packages included in this template.

% Uncomment and modify as needed:
% \begin{algorithm}[t]
% \caption{Your Algorithm Name}
% \label{alg:main}
% \begin{algorithmic}[1]
% \REQUIRE Input parameters
% \ENSURE Output
% \STATE Initialize variables
% \FOR{each iteration}
%     \STATE Perform operation
% \ENDFOR
% \RETURN result
% \end{algorithmic}
% \end{algorithm}

\subsection{Theoretical Analysis (Optional)}
\label{sec:theory}

[Theoretical Results] If your method has theoretical guarantees, state them here with proofs or proof sketches. You can defer full proofs to the appendix.

% Example:
% \begin{theorem}
% Under assumptions A1-A3, our method achieves...
% \end{theorem}
% \begin{proof}
% Proof sketch here. Full proof in Appendix~\ref{app:proofs}.
% \end{proof}

\subsection{Complexity Analysis (Optional)}
\label{sec:complexity}

[Complexity] Discuss time and space complexity if relevant. How does your method scale?
