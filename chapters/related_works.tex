
\section{Related Work}
\label{sec:related}
% ========================================
% RELATED WORK SECTION TEMPLATE
% ========================================
% This section positions your work within the broader research landscape.
% Guidelines:
% 1. Organize by themes/categories, not chronologically
% 2. Be comprehensive but concise
% 3. Clearly distinguish your work from prior work
% 4. Be respectful and accurate when describing others' work
% 5. Cite papers appropriately using \cite{} or \citep{}

% Optional: Include a comparison table
% \input{tables/comparison}

% Organize related work into coherent subsections by research theme

\xhdr{Research Area 1: [Name]}
[Theme Description] Provide a brief overview of this research direction. What is the main focus and how does it relate to your work?

[Prior Work Discussion] Discuss key papers in this area. Group related papers together and explain their approaches, contributions, and limitations. Be specific about what each work contributes and how it differs from yours.
% Example: "Several works have explored [approach], including \citet{paper1} who proposed [method1] and \citet{paper2} who extended this to [method2]. However, these approaches are limited by [specific limitation]."

\xhdr{Research Area 2: [Name]}
[Second Theme] Describe another related research direction. Continue organizing prior work thematically.

[Key Papers] Discuss relevant papers, highlighting similarities and differences with your work. Focus on:
- What problem they address
- Their approach and key innovations
- Their limitations or differences from your work
- How your work builds upon or differs from theirs

\xhdr{Research Area 3: [Name]}
[Additional Themes] Add more subsections as needed to cover all relevant related work areas.

% Some papers use bullet points to organize dense related work:
% \begin{itemize}
%     \item \textbf{Subarea 1:} Discussion of papers in this subarea~\citep{paper1, paper2}.
%     \item \textbf{Subarea 2:} Discussion of papers in this subarea~\citep{paper3, paper4}.
% \end{itemize}

\xhdr{Comparison with Our Work}
[Positioning] Conclude by clearly articulating how your work differs from and advances beyond prior work. What unique contribution does your paper make?

% Optional: Include a comparison table summarizing key differences
% Example: "Table~\ref{tab:comparison} compares our method with related approaches across key dimensions: [dimension1], [dimension2], and [dimension3]. Our method is the first to [unique capability]."

% Some conferences prefer related work near the end, others near the beginning.
% This template places it near the end (common for ML conferences).
% Move to Section 2 if your conference/advisor prefers early placement.
