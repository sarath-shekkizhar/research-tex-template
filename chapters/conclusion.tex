
\section{Conclusion}
\label{sec:conclusion}
% ========================================
% CONCLUSION SECTION TEMPLATE
% ========================================
% The conclusion should be concise (1-2 paragraphs typically).
% Structure:
% 1. Restate the problem and why it matters
% 2. Summarize your approach and key contributions
% 3. Highlight main findings from experiments
% 4. Discuss broader impact or future directions (optional)

% Paragraph 1: Problem and Solution Summary
[Problem Restatement] Briefly restate the problem addressed in this work and why it's important. This should echo your introduction but be more concise.

[Solution Summary] Summarize your approach in 1-2 sentences. What was the key insight or innovation?

% Paragraph 2: Contributions and Impact
[Key Contributions] Recap your main contributions and findings. What were the most important results from your experiments?

[Broader Impact / Future Work] Optionally discuss broader implications, limitations, or promising future directions. Keep this brief - the focus should be on what you accomplished, not what remains to be done.

% Example closing: "We believe this work opens new possibilities for [application area] and hope it inspires further research in [direction]."

% Note: Some papers include a separate "Limitations" subsection before the conclusion.
% Others integrate limitations into the conclusion or discussion section.
