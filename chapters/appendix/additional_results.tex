\section{Additional Experimental Results}
\label{app:additional-results}
% ========================================
% ADDITIONAL RESULTS APPENDIX TEMPLATE
% ========================================
% Use this section for experimental results that didn't fit in the main paper
% Common contents:
% - Extended results tables
% - Additional ablation studies
% - Results on additional datasets
% - Sensitivity analyses

\subsection{Extended Results on Additional Datasets}
\label{app:extended-results}

[Additional Dataset Results] Present results on datasets that couldn't fit in the main paper due to space constraints.

% Example table:
% \begin{table}[h]
% \centering
% \caption{Results on additional benchmark datasets}
% \label{tab:extended-results}
% \begin{tabular}{lccc}
% \toprule
% Method & Dataset A & Dataset B & Dataset C \\
% \midrule
% Baseline & 70.5 & 75.2 & 68.9 \\
% Ours & \textbf{75.8} & \textbf{79.1} & \textbf{73.2} \\
% \bottomrule
% \end{tabular}
% \end{table}

\subsection{Additional Ablation Studies}
\label{app:additional-ablations}

[Extended Ablations] Include more detailed ablation studies or analyses of design choices.

\subsection{Hyperparameter Sensitivity}
\label{app:hyperparameters}

[Parameter Analysis] Detailed analysis of hyperparameter choices and sensitivity.

% Example figure:
% \begin{figure}[h]
% \centering
% \includegraphics[width=0.7\linewidth]{figures/hyperparameter_sweep}
% \caption{Performance vs. hyperparameter values}
% \label{fig:hyperparameters}
% \end{figure}
