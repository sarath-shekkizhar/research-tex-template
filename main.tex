\documentclass{article}
\usepackage[T1]{fontenc}
\usepackage{inconsolata}

\usepackage{enumitem}
\usepackage{hyperref}


\newcommand{\theHalgorithm}{\arabic{algorithm}}

\usepackage[accepted]{icml2025}
% \usepackage[accepted]{icml2025_arxiv} % arxiv
%\usepackage[accepted]{icml2025} % if accepted
%\usepackage{icml2025_anon} % submission


\newcommand{\xhdr}[1]{{\noindent\bfseries #1}.}

\usepackage{amsmath}
\usepackage{amssymb}
\usepackage{mathtools}
\usepackage{amsthm}
\usepackage{graphicx}
\usepackage{booktabs}
\usepackage{subcaption}
\usepackage{colortbl}
\usepackage{makecell}
\usepackage{caption}
\usepackage{multirow}
\usepackage{listings}
\usepackage{cuted}
\usepackage{multicol}
\usepackage{titling}
\usepackage{manyfoot}
\usepackage{verbatim}
\usepackage{xspace}
\usepackage{marginnote}
\usepackage{xcolor}
\usepackage[capitalize,noabbrev]{cleveref}

\usepackage{tikz}
\newcommand*\circled[1]{\tikz[baseline=(char.base)]{
            \node[shape=circle,draw,inner sep=0.5pt] (char) {#1};}}

\theoremstyle{plain}
\newtheorem{theorem}{Theorem}[section]
\newtheorem{proposition}[theorem]{Proposition}
\newtheorem{lemma}[theorem]{Lemma}
\newtheorem{corollary}[theorem]{Corollary}
\theoremstyle{definition}
\newtheorem{definition}[theorem]{Definition}
\newtheorem{assumption}[theorem]{Assumption}
\theoremstyle{remark}
\newtheorem{remark}[theorem]{Remark}
\usepackage{tabularx}
\usepackage{quoting}
\newenvironment{ex}[1]
 {%
  \quoting[leftmargin=0.1cm,rightmargin=0.1cm]%
  \noindent\textcolor{blue!75}{} \itshape\ignorespaces
 }
 {\endquoting}
% ========================================
% CUSTOM COMMANDS
% ========================================

% Common abbreviations - keep these, they're universally useful
\newcommand{\ie}{\textit{i.e., }}
\newcommand{\eg}{\textit{e.g., }}
\newcommand{\etal}{\textit{et al.}}
\newcommand{\st}{\textit{s.t. }}
\newcommand{\etc}{\textit{etc.}}
\newcommand{\wrt}{\textit{w.r.t. }}
\newcommand{\cf}{\textit{cf. }}
\newcommand{\aka}{\textit{aka. }}

% Paper-specific commands - add your custom commands here
% Example: \newcommand{\methodname}{\textsc{YourMethod}\xspace}
% Example: \newcommand{\datasetname}{\texttt{YourDataset}\xspace}
% Example: \newcommand{\accuracy}{95.3\%\xspace}

\renewcommand{\thefootnote}{\fnsymbol{footnote}}
\usepackage{inconsolata} % Import modern font package
\lstset{
  basicstyle=\fontfamily{pcr}\selectfont\itshape\small, % Monospace + Italic + Small font
  frame=tb,                   
  breaklines=true,             
  breakatwhitespace=false,     
  linewidth=\linewidth,        
  numbers=left,                
  numberstyle=\tiny\color{gray}, 
  xleftmargin=10pt,            
  framexleftmargin=10pt,       
  tabsize=4,                   
  showstringspaces=false,      
  keywordstyle=\color{blue}\bfseries,  
  commentstyle=\color{gray},   
  stringstyle=\color{red},     
  aboveskip=10pt,              
  belowskip=10pt               
}
\usepackage{pifont}
\usepackage{xcolor}
\setlist[itemize]{leftmargin=10pt, itemsep=0pt, topsep=0pt, parsep=0pt}

\usepackage{pifont}
\usepackage{fontawesome5}

\definecolor{linkcolor}{rgb}{0.0, 0.0, 0.55} % dark blue
%\definecolor{linkcolor}{rgb}{0.4, 0.0, 0.6} % dark purple

\usepackage[textsize=tiny]{todonotes}

% ========================================
% PAPER TITLE AND AUTHORS
% ========================================
% Replace with your paper title and author information

\icmltitlerunning{Your Short Title for Running Head}

\begin{document}
\onecolumn  % For ICML format
\icmltitle{Your Paper Title Here: Replace with Your Actual Title}
\vspace{-10pt}
\begin{icmlauthorlist}
% Add your authors here
\icmlauthor{First Author}{inst1}
\icmlauthor{Second Author}{inst2}
\icmlauthor{Third Author}{inst1}\\
% Add more authors as needed
[0.1cm]
% Optional: Add project website/links
% \href{http://your-project-url.com}{\footnotesize {\textcolor{linkcolor}{\texttt{http://your-project-url.com}}}}
\end{icmlauthorlist}

% Define your institutions
\icmlaffiliation{inst1}{First Institution Name}
\icmlaffiliation{inst2}{Second Institution Name}
% Add more institutions as needed

% Corresponding author(s)
\icmlcorrespondingauthor{First Author}{first.author@institution.edu}
% \icmlcorrespondingauthor{Second Author}{second.author@institution.edu}

\icmlkeywords{Machine Learning, Deep Learning, Your Keywords}

\vskip 0.3in %% default: \vskip21.681pt

% Optional: Add a figure before the abstract if needed
% \begin{figure}[H]
%     \centering
%     \vspace{-20pt}
%     \includegraphics[width=0.865\linewidth]{figures/your_figure}
%     \vspace{-8pt}
%     \captionof{figure}{Your figure caption here describing the main contribution or overview.}
%     \label{fig:overview}
% \end{figure}

\begin{multicols}{2}  % Hank Hack

\printAffiliationsAndNotice{} % otherwise use the standard text.


\begin{abstract}
% ========================================
% ABSTRACT TEMPLATE
% ========================================
% Guidelines for writing a good abstract:
% 1. Start with the problem/motivation (1-2 sentences)
% 2. Introduce your approach/method (1-2 sentences)
% 3. Describe key technical contributions (1-2 sentences)
% 4. Present main results/findings (1-2 sentences)
% 5. Conclude with broader impact or significance (optional, 1 sentence)
%
% Keep it concise (typically 150-250 words for ML conferences)
% Avoid citations if possible
% Make it self-contained - readers should understand your contribution without reading the full paper

[Problem Statement] Start by describing the key problem or limitation in existing work that motivates your research. What gap are you addressing?

[Your Approach] Introduce your method or framework. Give it a clear name and briefly describe the key idea that makes it novel.

[Technical Contribution] Explain the main technical innovations or components of your approach. What makes it different from prior work?

[Experimental Results] Present your main quantitative results. Include key metrics and improvements over baselines. Be specific with numbers when possible.

[Broader Impact] Optionally, conclude with a sentence about the broader implications or potential applications of your work.

\end{abstract}

% Optional: Include a figure after the abstract if your format allows
% \input{latexfig/intro_examples}

\vspace{-15pt}
\section{Introduction}
% ========================================
% INTRODUCTION TEMPLATE
% ========================================
% A good introduction typically follows this structure:
% 1. Motivation & Context (1-2 paragraphs)
% 2. Problem Statement (1 paragraph)
% 3. Limitations of Existing Work (1 paragraph)
% 4. Your Approach Overview (1-2 paragraphs)
% 5. Key Contributions (1 paragraph, often bulleted)
% 6. Paper Organization (optional, 1 sentence)

% Paragraph 1: Motivation & Context
% Start broad - what is the general area and why is it important?
% Provide context for readers who may not be domain experts.
% Example: "Recent advances in [field] have enabled [capability], opening new possibilities for [application]."

[Motivation Paragraph] Describe the broader context and importance of your research area. Why should readers care about this problem domain? What recent developments or trends make this timely?

% Paragraph 2: Problem Statement
% Narrow down to the specific problem you're addressing.
% Make it clear what gap exists in current solutions.
% Example: "However, existing approaches face challenges when [specific scenario], limiting their applicability to [real-world use case]."

[Problem Statement] Identify the specific problem or limitation that your work addresses. What challenges do current methods face? Why is this problem difficult or important to solve?

% Paragraph 3: Limitations of Prior Work
% Discuss why existing approaches are insufficient.
% Be respectful but clear about what's missing.
% Cite relevant prior work appropriately.

[Prior Work Limitations] Explain what existing methods have tried and why they fall short. What key insight or capability is missing? Reference relevant prior work using \texttt{\textbackslash{}cite\{\}} commands.

% Paragraph 4-5: Your Approach
% Introduce your method/framework with a clear name.
% Explain the key insight or innovation that enables your approach.
% Reference figures if you have overview diagrams (e.g., Figure~\ref{fig:overview}).
% Describe the main technical components at a high level.

[Your Approach] Introduce your method or framework. Give it a clear, memorable name and explain the key innovation that distinguishes it from prior work. Describe the main components or stages of your approach. If you have an overview figure, reference it here.

% Paragraph 6: Key Contributions
% Clearly enumerate your contributions, often as a bulleted list.
% Be specific and concrete about what's novel.

[Contributions Summary] We make the following key contributions:
\begin{itemize}
    \item \textbf{Contribution 1:} A novel method/framework/algorithm for [specific task]. Briefly explain what makes it unique.
    \item \textbf{Contribution 2:} A new benchmark/dataset/evaluation for [specific purpose]. Describe its key characteristics.
    \item \textbf{Contribution 3:} Comprehensive experiments showing [X\% improvement] on [specific metrics] compared to baselines. Highlight key empirical findings.
    \item \textbf{Contribution 4 (optional):} Additional contributions such as theoretical analysis, user studies, open-source releases, etc.
\end{itemize}

% Optional: Paper Organization
% Some papers include a brief roadmap, others don't. Use your judgment.
% Example: "The rest of this paper is organized as follows: Section~\ref{sec:method} describes our approach, Section~\ref{sec:experiments} presents experimental results, and Section~\ref{sec:related} discusses related work."

% [Optional Paper Roadmap] The remainder of this paper is organized as follows: Section~\ref{sec:method} describes our method in detail, Section~\ref{sec:experiments} presents experimental results, Section~\ref{sec:related} discusses related work, and Section~\ref{sec:conclusion} concludes.


\section{Method}
\label{sec:method}
% ========================================
% METHOD SECTION TEMPLATE
% ========================================
% This section describes your approach in detail.
% Structure suggestions:
% 1. Problem Formulation (optional separate section or subsection)
% 2. Overview of your approach
% 3. Detailed description of each component
% 4. Algorithm pseudocode (if applicable)
% 5. Theoretical analysis (if applicable)

% Optional: Separate problem formulation section
\subsection{Problem Formulation}
\label{sec:formulation}

[Problem Setup] Define the problem formally. Introduce necessary notation and mathematical definitions. What are the inputs and outputs? What are the constraints or assumptions?

% Example notation:
% - Let $\mathcal{D} = \{(x_i, y_i)\}_{i=1}^N$ denote the dataset
% - Given input $x \in \mathcal{X}$, we aim to predict $y \in \mathcal{Y}$
% - Define the objective function as $\mathcal{L}(\theta) = ...$

\subsection{Method Overview}
\label{sec:method-overview}

[High-Level Description] Provide an overview of your approach before diving into details. Explain the key intuition and main steps. Reference any overview figures you have.

% Example: "Our method consists of three main stages: (1) preprocessing, (2) model training, and (3) inference optimization. Figure~\ref{fig:method} illustrates the complete pipeline."

\subsection{Component 1: [Name]}
\label{sec:component1}

[Detailed Component Description] Describe the first major component of your method. Include:
- Mathematical formulation
- Design choices and rationale
- Connections to prior work
- Any implementation details crucial for understanding

% Example subsections you might include:
% - Architecture design
% - Loss function
% - Training procedure
% - Optimization details

\subsection{Component 2: [Name]}
\label{sec:component2}

[Second Component] Describe additional components similarly. Break down complex methods into digestible parts.

\subsection{Algorithm}
\label{sec:algorithm}

[Pseudocode] If applicable, include algorithm pseudocode for clarity. Use the algorithm/algorithmic packages included in this template.

% Uncomment and modify as needed:
% \begin{algorithm}[t]
% \caption{Your Algorithm Name}
% \label{alg:main}
% \begin{algorithmic}[1]
% \REQUIRE Input parameters
% \ENSURE Output
% \STATE Initialize variables
% \FOR{each iteration}
%     \STATE Perform operation
% \ENDFOR
% \RETURN result
% \end{algorithmic}
% \end{algorithm}

\subsection{Theoretical Analysis (Optional)}
\label{sec:theory}

[Theoretical Results] If your method has theoretical guarantees, state them here with proofs or proof sketches. You can defer full proofs to the appendix.

% Example:
% \begin{theorem}
% Under assumptions A1-A3, our method achieves...
% \end{theorem}
% \begin{proof}
% Proof sketch here. Full proof in Appendix~\ref{app:proofs}.
% \end{proof}

\subsection{Complexity Analysis (Optional)}
\label{sec:complexity}

[Complexity] Discuss time and space complexity if relevant. How does your method scale?

\section{Experiments}
\label{sec:experiments}
% ========================================
% EXPERIMENTS SECTION TEMPLATE
% ========================================
% This section presents your empirical evaluation.
% Typical structure:
% 1. Experimental Setup (datasets, baselines, metrics, implementation)
% 2. Main Results
% 3. Ablation Studies
% 4. Additional Analysis (e.g., case studies, error analysis, qualitative results)

\subsection{Experimental Setup}
\label{sec:setup}

[Setup Overview] Provide an overview of your experimental evaluation. What research questions are you investigating?

% Optional: List research questions
% \begin{itemize}
%     \item \textbf{RQ1:} How does our method compare to state-of-the-art baselines?
%     \item \textbf{RQ2:} What components are critical to performance?
%     \item \textbf{RQ3:} How does the method generalize to different settings?
% \end{itemize}

\subsubsection{Datasets}
\label{sec:datasets}

[Dataset Description] Describe the datasets used for evaluation. Include:
- Dataset names and sources
- Size and statistics
- Train/validation/test splits
- Any preprocessing steps
- Why these datasets are appropriate for your task

% Example table for dataset statistics:
% \begin{table}[h]
% \centering
% \caption{Dataset statistics}
% \label{tab:datasets}
% \begin{tabular}{lccc}
% \toprule
% Dataset & Train & Val & Test \\
% \midrule
% Dataset1 & 10K & 1K & 2K \\
% Dataset2 & 50K & 5K & 10K \\
% \bottomrule
% \end{tabular}
% \end{table}

\subsubsection{Baselines}
\label{sec:baselines}

[Baseline Methods] Describe the baseline methods you compare against:
- State-of-the-art methods from prior work
- Simpler alternatives or ablations
- Why each baseline is relevant
- Implementation details and hyperparameters for baselines

\subsubsection{Evaluation Metrics}
\label{sec:metrics}

[Metrics] Define the evaluation metrics used:
- What metrics and why they're appropriate
- How they're computed
- Any standard benchmarks or protocols followed

\subsubsection{Implementation Details}
\label{sec:implementation}

[Technical Details] Provide sufficient implementation details for reproducibility:
- Model architecture specifics
- Hyperparameters and how they were selected
- Training procedures (optimizer, learning rate, batch size, epochs, etc.)
- Hardware used (GPUs, compute time)
- Any tricks or techniques that were important
- Links to code repositories if available

\subsection{Main Results}
\label{sec:main-results}

[Primary Findings] Present your main experimental results. Typically includes:
- Comparison table with baselines
- Statistical significance testing if applicable
- Discussion of key findings

% Example results table:
% \begin{table}[t]
% \centering
% \caption{Main results on benchmark datasets. Best results in \textbf{bold}.}
% \label{tab:main-results}
% \begin{tabular}{lcccc}
% \toprule
% Method & Dataset1 & Dataset2 & Dataset3 & Avg \\
% \midrule
% Baseline1 & 75.2 & 82.1 & 68.5 & 75.3 \\
% Baseline2 & 78.5 & 84.2 & 71.2 & 78.0 \\
% \textbf{Ours} & \textbf{82.1} & \textbf{87.3} & \textbf{75.8} & \textbf{81.7} \\
% \bottomrule
% \end{tabular}
% \end{table}

[Results Discussion] Discuss the main results:
- How your method compares to baselines
- Statistical significance of improvements
- Where your method excels and where it struggles
- Insights from the results

\subsection{Ablation Studies}
\label{sec:ablations}

[Component Analysis] Conduct ablation studies to understand which components are critical:
- Remove or modify individual components
- Quantify their contribution
- Justify design choices

% Example ablation table:
% \begin{table}[h]
% \centering
% \caption{Ablation study results}
% \label{tab:ablations}
% \begin{tabular}{lc}
% \toprule
% Model Variant & Performance \\
% \midrule
% Full Model & 82.1 \\
% \quad w/o Component A & 79.3 \\
% \quad w/o Component B & 80.1 \\
% \quad w/ Alternative C & 81.5 \\
% \bottomrule
% \end{tabular}
% \end{table}

\subsection{Additional Analysis}
\label{sec:analysis}

[Further Investigation] Include additional analysis that provides insights:

\subsubsection{Sensitivity Analysis}
[Parameter Sensitivity] How sensitive is your method to hyperparameters?

\subsubsection{Qualitative Analysis}
[Case Studies] Include examples or case studies that illustrate your method's behavior.

% Reference figures with qualitative results:
% See Figure~\ref{fig:qualitative} for example outputs.

\subsubsection{Error Analysis}
[Failure Cases] When does your method fail? What are common error patterns?

\subsection{Computational Cost}
\label{sec:cost}

[Efficiency] Discuss computational requirements:
- Training time
- Inference time
- Memory usage
- Comparison with baselines


\section{Related Work}
\label{sec:related}
% ========================================
% RELATED WORK SECTION TEMPLATE
% ========================================
% This section positions your work within the broader research landscape.
% Guidelines:
% 1. Organize by themes/categories, not chronologically
% 2. Be comprehensive but concise
% 3. Clearly distinguish your work from prior work
% 4. Be respectful and accurate when describing others' work
% 5. Cite papers appropriately using \cite{} or \citep{}

% Optional: Include a comparison table
% \input{tables/comparison}

% Organize related work into coherent subsections by research theme

\xhdr{Research Area 1: [Name]}
[Theme Description] Provide a brief overview of this research direction. What is the main focus and how does it relate to your work?

[Prior Work Discussion] Discuss key papers in this area. Group related papers together and explain their approaches, contributions, and limitations. Be specific about what each work contributes and how it differs from yours.
% Example: "Several works have explored [approach], including \citet{paper1} who proposed [method1] and \citet{paper2} who extended this to [method2]. However, these approaches are limited by [specific limitation]."

\xhdr{Research Area 2: [Name]}
[Second Theme] Describe another related research direction. Continue organizing prior work thematically.

[Key Papers] Discuss relevant papers, highlighting similarities and differences with your work. Focus on:
- What problem they address
- Their approach and key innovations
- Their limitations or differences from your work
- How your work builds upon or differs from theirs

\xhdr{Research Area 3: [Name]}
[Additional Themes] Add more subsections as needed to cover all relevant related work areas.

% Some papers use bullet points to organize dense related work:
% \begin{itemize}
%     \item \textbf{Subarea 1:} Discussion of papers in this subarea~\citep{paper1, paper2}.
%     \item \textbf{Subarea 2:} Discussion of papers in this subarea~\citep{paper3, paper4}.
% \end{itemize}

\xhdr{Comparison with Our Work}
[Positioning] Conclude by clearly articulating how your work differs from and advances beyond prior work. What unique contribution does your paper make?

% Optional: Include a comparison table summarizing key differences
% Example: "Table~\ref{tab:comparison} compares our method with related approaches across key dimensions: [dimension1], [dimension2], and [dimension3]. Our method is the first to [unique capability]."

% Some conferences prefer related work near the end, others near the beginning.
% This template places it near the end (common for ML conferences).
% Move to Section 2 if your conference/advisor prefers early placement.


\section{Conclusion}
\label{sec:conclusion}
% ========================================
% CONCLUSION SECTION TEMPLATE
% ========================================
% The conclusion should be concise (1-2 paragraphs typically).
% Structure:
% 1. Restate the problem and why it matters
% 2. Summarize your approach and key contributions
% 3. Highlight main findings from experiments
% 4. Discuss broader impact or future directions (optional)

% Paragraph 1: Problem and Solution Summary
[Problem Restatement] Briefly restate the problem addressed in this work and why it's important. This should echo your introduction but be more concise.

[Solution Summary] Summarize your approach in 1-2 sentences. What was the key insight or innovation?

% Paragraph 2: Contributions and Impact
[Key Contributions] Recap your main contributions and findings. What were the most important results from your experiments?

[Broader Impact / Future Work] Optionally discuss broader implications, limitations, or promising future directions. Keep this brief - the focus should be on what you accomplished, not what remains to be done.

% Example closing: "We believe this work opens new possibilities for [application area] and hope it inspires further research in [direction]."

% Note: Some papers include a separate "Limitations" subsection before the conclusion.
% Others integrate limitations into the conclusion or discussion section.

\newpage
\section*{Acknowledgments}
% ========================================
% ACKNOWLEDGEMENTS TEMPLATE
% ========================================
% Thank people and organizations that contributed to your work.
% Common acknowledgements include:
% - Funding sources (required by most grants)
% - Colleagues who provided feedback
% - Computing resources
% - Dataset providers
% - Anonymous reviewers (if this is a journal or revised version)

% [Acknowledgement Text]
We thank [names or groups] for their valuable feedback and suggestions. We are grateful to [institution/organization] for providing computational resources.

We also gratefully acknowledge the support of [funding agency] under grant [grant number], [additional funding sources].

% Example structure:
% We thank [Person 1], [Person 2], and [Person 3] for helpful discussions and feedback on early drafts.
% We are grateful to [Institution] for providing access to computational resources.
% This work was supported by NSF Grant IIS-1234567, a gift from [Company], and [other funding sources].

% Note: Some conferences request acknowledgements be anonymized for double-blind review.
% In such cases, comment out this section for submission and add it back after acceptance.

\section*{Impact Statement}
% ========================================
% IMPACT STATEMENT TEMPLATE
% ========================================
% Some conferences (like NeurIPS, ICML) require an impact statement discussing
% the broader societal implications of your work.
%
% Structure:
% - Positive impacts and potential applications
% - Potential negative consequences or risks
% - Ethical considerations
% - Mitigation strategies for identified risks
%
% Be thoughtful and honest. Avoid being overly promotional or dismissive of risks.

% You can remove this section if your target conference doesn't require it.

\label{sec:impact}

% Option 1: Minimal statement (if impacts are limited)
This paper presents work whose goal is to advance the field of [your field]. There are many potential societal consequences of our work, none of which we feel must be specifically highlighted here.

% Option 2: Detailed statement (use this structure if your work has significant societal implications)
% [Positive Impacts]
% This work aims to [primary goal], which has potential applications in [domains].
% The proposed approach could benefit [stakeholders] by [specific benefits].
% For example, [concrete positive application].

% [Potential Risks]
% However, like any [technology/method], our approach could potentially be misused for [negative applications].
% We acknowledge risks including [specific risk 1], [specific risk 2], and [specific risk 3].

% [Ethical Considerations]
% We have taken steps to address ethical concerns:
% - [Measure 1: e.g., evaluating for bias across demographic groups]
% - [Measure 2: e.g., implementing safeguards against misuse]
% - [Measure 3: e.g., ensuring data privacy and consent]

% [Data and Privacy Considerations]
% If your work involves human subjects data:
% Our study involved [number] participants recruited through [platform].
% We ensured ethical treatment by: [IRB approval if applicable], [informed consent], [fair compensation], [privacy protections].
% Participants were compensated [amount] for [time], which exceeds [relevant minimum wage].

% [Mitigation Strategies]
% To mitigate potential harms, we recommend:
% - [Specific recommendation 1]
% - [Specific recommendation 2]
% - [Release strategy: e.g., providing access through API with monitoring, releasing only evaluation code, etc.]

% [Broader Perspective]
% We believe this work [positive closing statement about advancing the field responsibly].


\bibliography{main}
\bibliographystyle{icml2025}

\end{multicols}
\newpage
\onecolumn

% ========================================
% APPENDIX
% ========================================
\appendix
% Include your appendix sections here
% Uncomment the ones you need and add your own

\section{Additional Experimental Results}
\label{app:additional-results}
% ========================================
% ADDITIONAL RESULTS APPENDIX TEMPLATE
% ========================================
% Use this section for experimental results that didn't fit in the main paper
% Common contents:
% - Extended results tables
% - Additional ablation studies
% - Results on additional datasets
% - Sensitivity analyses

\subsection{Extended Results on Additional Datasets}
\label{app:extended-results}

[Additional Dataset Results] Present results on datasets that couldn't fit in the main paper due to space constraints.

% Example table:
% \begin{table}[h]
% \centering
% \caption{Results on additional benchmark datasets}
% \label{tab:extended-results}
% \begin{tabular}{lccc}
% \toprule
% Method & Dataset A & Dataset B & Dataset C \\
% \midrule
% Baseline & 70.5 & 75.2 & 68.9 \\
% Ours & \textbf{75.8} & \textbf{79.1} & \textbf{73.2} \\
% \bottomrule
% \end{tabular}
% \end{table}

\subsection{Additional Ablation Studies}
\label{app:additional-ablations}

[Extended Ablations] Include more detailed ablation studies or analyses of design choices.

\subsection{Hyperparameter Sensitivity}
\label{app:hyperparameters}

[Parameter Analysis] Detailed analysis of hyperparameter choices and sensitivity.

% Example figure:
% \begin{figure}[h]
% \centering
% \includegraphics[width=0.7\linewidth]{figures/hyperparameter_sweep}
% \caption{Performance vs. hyperparameter values}
% \label{fig:hyperparameters}
% \end{figure}

\section{Implementation Details}
\label{app:implementation}
% ========================================
% IMPLEMENTATION DETAILS APPENDIX TEMPLATE
% ========================================
% Provide detailed implementation information for reproducibility
% This section complements the brief implementation section in the main paper

\subsection{Architecture Details}
\label{app:architecture}

[Detailed Architecture] Provide complete architecture specifications:
- Layer dimensions
- Activation functions
- Normalization techniques
- Regularization methods
- Any architectural tricks or modifications

% Example: Include architecture diagram or detailed specification table
% \begin{table}[h]
% \centering
% \caption{Detailed model architecture}
% \label{tab:architecture}
% \begin{tabular}{lcc}
% \toprule
% Layer & Output Shape & Parameters \\
% \midrule
% Input & (batch, 224, 224, 3) & - \\
% Conv1 & (batch, 112, 112, 64) & 9,408 \\
% ... & ... & ... \\
% \bottomrule
% \end{tabular}
% \end{table}

\subsection{Training Details}
\label{app:training}

[Training Configuration] Complete training procedure details:
- Optimizer: [name, settings]
- Learning rate schedule: [strategy, parameters]
- Batch size: [value]
- Number of epochs: [value]
- Gradient clipping: [if applicable]
- Early stopping criteria: [if applicable]
- Hardware: [GPU/TPU specifications]
- Training time: [approximate duration]
- Random seeds: [values used]

\subsection{Data Preprocessing}
\label{app:preprocessing}

[Data Pipeline] Detailed description of data preprocessing:
- Normalization procedures
- Augmentation techniques
- Filtering criteria
- Train/val/test split procedures
- Any special handling of edge cases

\subsection{Hyperparameters}
\label{app:hyperparameters-table}

[Complete Hyperparameter List] Comprehensive list of all hyperparameters:

% \begin{table}[h]
% \centering
% \caption{Complete hyperparameter settings}
% \label{tab:hyperparameters}
% \begin{tabular}{lc}
% \toprule
% Hyperparameter & Value \\
% \midrule
% Learning rate & 1e-4 \\
% Batch size & 32 \\
% Weight decay & 1e-5 \\
% ... & ... \\
% \bottomrule
% \end{tabular}
% \end{table}

\subsection{Code and Resources}
\label{app:code}

[Reproducibility Resources]
- Code repository: [URL]
- Pre-trained models: [URL]
- Datasets: [URL]
- Dependencies: [list key libraries and versions]

% Example:
% Our code is available at \url{https://github.com/username/project}.
% Pre-trained models can be downloaded from \url{https://...}.
% All experiments were conducted using PyTorch 2.0, Python 3.10, and CUDA 11.8.

\section{Theoretical Proofs}
\label{app:proofs}
% ========================================
% PROOFS APPENDIX TEMPLATE
% ========================================
% Use this section for detailed proofs of theorems/lemmas from the main paper
% Defer lengthy proofs here to keep the main paper readable

% Reference theorems from main paper using labels

\subsection{Proof of Theorem X}
\label{app:proof-theorem-x}

% State the theorem again for reader convenience
\begin{theorem}[Restatement of Theorem X]
[Theorem statement goes here]
\end{theorem}

\begin{proof}
[Detailed proof steps]

% Structure your proof with clear steps
\noindent \textbf{Step 1: [Description]}
[Proof content for step 1]

\noindent \textbf{Step 2: [Description]}
[Proof content for step 2]

% Continue with additional steps as needed
\end{proof}

\subsection{Proof of Lemma Y}
\label{app:proof-lemma-y}

\begin{lemma}[Restatement of Lemma Y]
[Lemma statement]
\end{lemma}

\begin{proof}
[Proof content]
\end{proof}

% Add additional proofs as needed


% Additional appendix sections you might need:
% \input{chapters/appendix/dataset_details}
% \input{chapters/appendix/additional_ablations}
% \input{chapters/appendix/qualitative_examples}
% \input{chapters/appendix/limitations}
% \input{chapters/appendix/broader_impact}

\end{document}
